\section{Reakcióerők meghatározása}

\subsection{Méretarányos ábra}
\begin{center}
        \strucguide
        \begin{tikzpicture}
                \strucframe
                \strucforces
                \strucrestraints
                \strucsharedforces
        \end{tikzpicture}
\end{center}

A \textcolor{olive}{kényszerek} ábrázolásra kerültek az esetleges félreértések elkerülése érdekében.
Egy darab külső \textcolor{blue}{erő} jelenik meg illetve $C$ és $B$ között pedig egy \textcolor{orange}{megoszló erő}.


\begin{align*}
	&a =  \pnum{\a}\si{[m]} \\
        &b =  \pnum{\b}\si{[m]} \\
        &c =  \pnum{\c}\si{[m]}
\end{align*}

\begin{align*}
        k =  \pnum{\k}\si{[kN]} \\
        p =  \pnum{\p}\si{[kN]} \\
        F =  \pnum{\F}\si{[kN]} \\
\end{align*}

\subsection{SZTÁ}

\begin{center}
        \strucguide
        \begin{tikzpicture}
                \strucframe
                \strucforces
                \strucsharedforces
                \strucrestrainforces
        \end{tikzpicture}
\end{center}


\break

\subsection{Egyensúlyi egyenletek}

\begin{align*}
	F_\text{p}(x) &= p \times x \\
	{F_\text{p}}_{\text{max}} &= F_\text{p}(b+c) = \pnum{\Fp} \si{[kN]}
\end{align*}

\begin{align*}
	&\sum{F_\text{x}} := 0 = A_\text{x} + B_\text{x} \\
	&\sum{F_\text{y}} := 0 = A_\text{y} + F - {F_\text{p}}_{\text{max}}  \\
	&\sum{M^{_\text{A}}_\text{z}} := 0 = M_\text{A} - F_\text{p}(a + \frac{b+c}{2}) + F \times (a+b) - B_\text{x} \times k
\end{align*}

$$
A_\text{y} = {F_\text{p}}_{\text{max}} - F = 1.5 \si{[kN]}
$$

3 ismeretlen és 2 egyenlet maradt tehát a szerkezetet részekre kell bontanunk.

\subsection{Részek vizsgálata}

A \textcolor{orange}{$\mathbf{C}$} pontban kettévágva a rácsszerkezetet részenként vizsgálhatom (így ezen pont mindkét ábrának része). Az ebben a pontban ébredő belső reakcióerőket a két részen ellentétesen veszem fel \textbf{Newton III. törvénye} (hatás-ellenhatás) miatt.

\begin{multicols}{2}

\subsubsection{}
\begin{center}
        \begin{tikzpicture}
                \strucframeone
                \strucrestrainforcesone
                \foreach \point in {C} {
                        \fill[thick, cyan] (\point) circle[radius=4pt];
                }
                \struccutforcesone
        \end{tikzpicture}
\end{center}

\begin{align*}
	&\sum{F_\text{x}} := 0 = A_\text{x} + C_\text{x} \\
	&\sum{F_\text{y}} := 0 = A_\text{y} + C_\text{y} \\
	&\sum{M^{_\text{C}}_\text{z}} := 0 = -A_\text{y} \times a + M_\text{A}
\end{align*}

\begin{align*}
	&C_\text{y} = -A_\text{y} = \pnum{\Cy} \\
	&M_\text{A} = A_\text{y} \times a = \pnum{\Ma} \\
	&A_\text{x} = -C_\text{x} = \Ax
\end{align*}

\subsubsection{}
\begin{center}
        \begin{tikzpicture}
                \strucframetwo
                \strucforcestwo
                \strucsharedforcestwo
                \strucrestrainforcestwo
                \foreach \point in {C} {
                        \fill[thick, cyan] (\point) circle[radius=4pt];
                }
                \struccutforcestwo
        \end{tikzpicture}
\end{center}

\begin{align*}
	&\sum{F_\text{x}} := 0 = - C_\text{x} + B_\text{x} \\
	&\sum{F_\text{y}} := 0 = - C_\text{y} - F_\text{p} + F \\
	&\sum{M^{_\text{C}}_\text{z}} := 0 = -{F_\text{p}}_\text{max} \times \frac{b+c}{2} - B_\text{x} \times k + F \times b
\end{align*}

\begin{align*}
	&B_\text{x} = \frac{F \times b - {F_\text{p}}_\text{max} \times \frac{b+c}{2}}{k} = \pnum{\Bx} \\
	&C_\text{x} = B_\text{x} = \pnum{\Cx}
\end{align*}

\end{multicols}
