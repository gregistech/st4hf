\section[\texorpdfstring{Igénybevételi ábrák}{Igénybevételi ábrák}]%
{Igénybevételi ábrák\protect\footnote{A konstant 0 függvények nem kerültek ábrázolásra.}}

\pgfmathsetmacro{\framewidth}{.7mm}

\subsection{Vízszintes rúd}

% FIXME: equations dont care about variables
\subsubsection{Parabola szerkesztése}


A véges differenciák módszere itt abban nyújt segítséget hogy a derivált értékét numerikusan közelíthetjük, különösen ha a függvény nem analitikus hanem diszkrét pontokból áll. Például ha a parabola értékei $f(x)$ egy adott $x_0$ pont környezetében ismertek akkor a deriváltat közelíthetjük előre (előre differencia), hátra (hátra differencia), vagy közép differenciával.

\vskip 20pt

\begin{itemize}
	\item \textbf{Előre differencia:}
		$f'(x_0) \approx \frac{f(x_0 + h) - f(x_0)}{h}$

	\item \textbf{Hátra differencia:}
		$f'(x_0) \approx \frac{f(x_0) - f(x_0 - h)}{h}$

	\item \textbf{Közép differencia:}
		$f'(x_0) \approx \frac{f(x_0 + h) - f(x_0 - h)}{2h}$
\end{itemize}

\vskip 20pt

A közép differencia módszer különösen pontos mert elsőrendű hibákat eliminál és így gyorsabban közelíti az analitikus deriváltat. Ezekkel a módszerekkel a parabola pontjaira és érintőire numerikus módon, a véges differenciák segítségével is jó közelítést adhatunk ami különösen hasznos diszkrét adathalmazok vagy numerikus modellek esetében.

A megfelelő szélsőértékek megállapítása után ezen pontokban kiszámolhatjuk a parabola érintőinek meredekségét (egyenletük egyszerűbb ábrázolás érdekében). Ezt az ${M'}_\text{h} = -V$ összefüggéssel egyszerűen megtehetjük.

Megnézve hol a derivált zérushelye (parabolánál mindig a középvonalra esik) abban az $x_\text{0}$-ban kiszámítjuk a függvény értékét.

Ezekkel az adatokkal felvértezve már egy egész jó közelítést adhatunk a parabola pontjaira.

\vskip 100pt

\begin{center}
\begin{tabularx}{\textwidth}{X | X}
	\makecell{$2.5x^2 - 4.5x + 1.8$} & \makecell{$2.5x^2 - 4.5x + 1.8$} \\
	\hline
	\makecell{
		$\begin{array}{rl}
	    & x_\text{1} = 0.6 \\
	    & x_\text{2} = 1.2 \\
	    \hline
	    & m_\text{1} = {M'}_\text{h}(x)|_{x=x_\text{1}} = -\frac{3}{2} \\
			\Rightarrow & m_\text{1} : -\frac{3}{2} \times x + 0.9 \\
				    & m_\text{2} = {M'}_\text{h}(x)|_{x=x_\text{2}} = \frac{3}{2} \\
			\Rightarrow & m_\text{2} : \frac{3}{2} \times x - 1.8 \\
				    & y_\text{0} = M_\text{h}(\frac{x_\text{1} + x_\text{2}}{2}) = -0.225 \\
	    \hline
				    &M'_\text{h}(x^*) = 0 \\
				    &x^* = 0.9 \\
				    &M_\text{h}(x^*) = -0.225
		\end{array}$
	} & 
	\makecell{
		$\begin{array}{rl}
	    & x_\text{1} = 1.2 \\
	    & x_\text{2} = 2.2 \\
	    \hline
	    & m_\text{1} = {M'}_\text{h}(x)|_{x=x_\text{1}} = -\frac{5}{2} \\
			\Rightarrow & m_\text{1} : -\frac{5}{2} \times x + 3 \\
				    & m_\text{2} = {M'}_\text{h}(x)|_{x=x_\text{2}} = \frac{5}{2} \\
			\Rightarrow & m_\text{2} : \frac{5}{2} \times x - 5.5 \\
				    & y_\text{0} = M_\text{h}(\frac{x_\text{1} + x_\text{2}}{2}) = -0.46875 \\
	    \hline
				    &M'_\text{h}(x_\text{0}) = 0 \\
				    &x_\text{0} = 1.7
		\end{array}$
	}
\end{tabularx}
\end{center}

\break

\subsubsection{Igénybevételi ábrák}

\begin{center}
        \begin{tikzpicture}
		\draw[line width = 0.2mm, arrows = {Latex-Latex}] (A)++(0,-2) -- ($ (C) + (0,-2) $) node[midway, below] {a};
		\draw[line width = 0.2mm, arrows = {Latex-Latex}] (C)++(0,-2) -- ($ (F) + (0,-2) $) node[midway, below] {b};
		\draw[line width = 0.2mm, arrows = {Latex-Latex}] (F)++(0,-2) -- ($ (B) + (0,-\ks-2) $) node[midway, below] {c};

		\draw[line width = \framewidth, arrows = {-}] (A) -- ($ (B) + (0, -\ks) $);
		\strucforces
		\strucsharedforces
		\strucrestrainforcesa

		\draw[thick,->,gray] (0,0) -- ++(2,0) node[anchor=north west] {$x_\text{1}$};
		\draw[thick,->,gray] (0,0) -- ++(0,2) node[anchor=south east] {y};

		\pgfmathsetmacro{\drawlength}{20}
		\draw[line width = 0.1mm, opacity=.3] (\as + \bs + \cs ,0) -- ++(0,-\drawlength);
		\foreach \point in {A, C, F} {
			\draw[line width = 0.1mm, opacity=.3] (\point) -- ++(0,-\drawlength);
		}
		
		\begin{axis}[
			xlabel=$x_\text{1} \si{[m]}$, ylabel=$N(x_\text{1}) \si{[kN]}$,
			width=15.2cm,
			height=5cm,
			xmin=0, xmax=\a + \b + \c,
			ymin=-7, ymax=0,
			ytick={0, ..., -5, -6.25},
			xtick={\a, \a + \b, \a + \b + \c},
			axis lines = left,
			axis x line = top,
			at={(0cm,-8cm)}
			]
			\addplot[name path=A,
				domain=0:\a + \b + \c, 
				smooth, 
				line width=2pt, 
				orange, 
				] {-6.25};
			\path [name path=B]
				(\pgfkeysvalueof{/pgfplots/xmin},0) --
				(\pgfkeysvalueof{/pgfplots/xmax},0);
			\addplot [pattern=vertical lines, pattern color=orange] fill between [of=A and B];
		\end{axis}

		\begin{axis}[
			xlabel=$x_\text{1} \si{[m]}$, ylabel=$V(x_\text{1}) \si{[kN]}$,
			width=15.2cm,
			height=5cm,
			xmin=0, xmax=\a + \b + \c,
			ymax=3,
			xtick={\a, \a + \b, \a + \b + \c},
			ytick={0, 1.5, -1.5, 2.5},
			axis lines = left,
			axis x line = middle,
			every axis x label/.style={
				at={(ticklabel* cs:1)},
				anchor=west,
			},
			at={(0cm,-12cm)}
			]
			\fill[thick, cyan] (1.2, 2.5) circle[radius=3pt] node[anchor=south west] {\tiny $V(x_\text{V})$};
			\addplot[name path=A,
				domain=0:\a, 
				smooth, 
				line width=2pt, 
				olive, 
				] {1.5};
			\addplot[name path=B,
				domain=\a:\a + \b, 
				smooth, 
				line width=2pt, 
				olive, 
				] {-5*x+4.5};
			\addplot[name path=C,
				domain=\a + \b:\a + \b + \c, 
				smooth, 
				line width=2pt, 
				olive, 
				] {-5*x+8.5};
			\path [name path=AG]
				(\pgfkeysvalueof{/pgfplots/xmin},0) --
				(\a,0);
			\path [name path=BG]
				(\a,0) --
				(\a + \b,0);
			\path [name path=CG]
				(\a + \b,0) --
				(\a + \b + \c,0);
			\addplot [pattern=vertical lines, pattern color=olive] fill between [of=A and AG];
			\addplot [pattern=vertical lines, pattern color=olive] fill between [of=B and BG];
			\addplot [pattern=vertical lines, pattern color=olive] fill between [of=C and CG];
		\end{axis}

		\begin{axis}[
			xlabel=$x_\text{1} \si{[m]}$, ylabel=$M_h(x_\text{1}) \si{[kNm]}$,
			width=15.2cm,
			height=8cm,
			xmin=0, xmax=\a + \b + \c,
			ymin=-1.5, ymax=1.2,
			xtick={\a, \a + \b, \a + \b + \c, .9},
			ytick={0, -0.225, -0.445, -0.625, -1.25, .9},
			axis lines = left,
			axis x line = middle,
			every axis x label/.style={
				at={(ticklabel* cs:1)},
				anchor=west,
			},
			at={(0cm,-19cm)}
			]
			\addplot[name path=A,
				domain=0:\a, 
				smooth, 
				line width=2pt, 
				blue, 
				] {-1.5*x+0.9};
			\addplot[
				domain=0:\a, 
				dotted, 
				line width=1pt, 
				black, 
				] {.9};

			\addplot[name path=B,
				domain=\a:\a + \b, 
				smooth, 
				line width=2pt, 
				blue, 
				] {2.5*x^2-4.5*x+1.8};
			\addplot[
				domain=\a:\a + \b, 
				dotted, 
				line width=1pt, 
				black, 
				] {-(3/2)*x+0.9};
			\addplot[
				domain=\a:\a + \b, 
				dotted, 
				line width=1pt, 
				black, 
				] {(3/2)*x-1.8};

			\fill[thick, cyan] (0, 0.9) circle[radius=3pt] node[anchor=south west] {\tiny $M_\text{h}(x_\text{$M_\text{h}$})$};
			\node[circle,fill,inner sep=2pt] at (axis cs:0.6,0) {};
			\node[circle,fill,inner sep=2pt] at (axis cs:1.2,0) {};
			\node[circle,fill,inner sep=2pt] at (axis cs:0.9,-0.45) {};
			\fill[thick, purple] (0.9, -0.225) circle[radius=3pt] node[below] {\tiny $M_\text{h}(x^*)$};
			\addplot [
				dotted, 
				line width=1pt, 
				black] coordinates {(.9, -1) (.9, 3)};
			\addplot [
				smooth, 
				line width=1pt, 
				black] coordinates {(.8, -0.225) (1, -0.225)};

			\addplot[name path=C,
				domain=\a + \b:\a + \b + \c, 
				smooth, 
				line width=2pt, 
				blue, 
				] {2.5*x^2-8.5*x+6.6};
			\addplot[
				domain=\a + \b:\a + \b + \c, 
				dotted, 
				line width=1pt, 
				black, 
				] {-(5/2)*x+3};
			\addplot[
				domain=\a + \b:\a + \b + \c, 
				dotted, 
				line width=1pt, 
				black, 
				] {(5/2)*x-5.5};
			\node[circle,fill,inner sep=2pt] at (axis cs:1.7,-0.625) {};
			\node[circle,fill,inner sep=2pt] at (axis cs:1.7,-1.25) {};
			\addplot [
				smooth, 
				line width=1pt, 
				black] coordinates {(1.6, -0.625) (1.8, -0.625)};

			
			\path [name path=AG]
				(\pgfkeysvalueof{/pgfplots/xmin},0) --
				(\a,0);
			\path [name path=BG]
				(\a,0) --
				(\a + \b,0);
			\path [name path=CG]
				(\a + \b,0) --
				(\a + \b + \c,0);
			\addplot [pattern=vertical lines, pattern color=blue] fill between [of=A and AG];
			\addplot [pattern=vertical lines, pattern color=blue] fill between [of=B and BG];
			\addplot [pattern=vertical lines, pattern color=blue] fill between [of=C and CG];
		\end{axis}

        \end{tikzpicture}
\end{center}

\break

\subsection{Függöleges rúd}

\subsubsection{Igénybevételi ábrák}
\begin{center}
        \begin{tikzpicture}
		\pgfmathsetmacro{\drawlength}{20}
		\draw[line width = 0.1mm, opacity=.3] (A) -- ++(0, -\drawlength);
		\draw[line width = 0.1mm, opacity=.3] (B)++(0,-\ks) -- ++(0, -\drawlength);

		\draw[line width = 0.2mm, arrows = {Latex-Latex}] (A)++(0,-2) -- ($ (B) + (0,-\ks-2) $) node[midway, below] {k};

		\draw[line width = \framewidth, arrows = {-}] (A) -- ($ (B) + (0, -\ks) $);
		\draw[line width = 0.1mm] (\as + \bs + \cs, 0) -- ++(0,-2);

		\draw[line width = .8mm, arrows = {-Stealth}, olive] (\as + \bs + \cs, -\Bxs) -- ++(0,\Bxs) node[midway, above, anchor=south east] {$B_\text{x}$};

		\draw[thick,->,gray] (0,0) -- ++(2,0) node[anchor=north west] {$x_\text{2}$};
		\draw[thick,->,gray] (0,0) -- ++(0,2) node[anchor=south east] {y};

		\begin{axis}[
			xlabel=$x_\text{2} \si{[m]}$, ylabel=$V(x_\text{2}) \si{[kN]}$,
			width=15.2cm,
			height=5cm,
			xmin=0, xmax=\k,
			ymin=0, ymax=6.25,
			ytick={0, ..., 5, 6.25},
			xtick={\k},
			axis lines = left,
			axis x line = middle,
			every axis x label/.style={
				at={(ticklabel* cs:1)},
				anchor=west,
			},
			at={(0cm,-9cm)}
			]
			\addplot[name path=A,
				domain=0:\k, 
				smooth, 
				line width=2pt, 
				orange, 
			] {6.25};
			\path [name path=B]
				(\pgfkeysvalueof{/pgfplots/xmin},0) --
				(\pgfkeysvalueof{/pgfplots/xmax},0);
			\addplot [pattern=vertical lines, pattern color=orange] fill between [of=A and B];
		\end{axis}

		\begin{axis}[
			xlabel=$x_\text{2} \si{[m]}$, ylabel=$M_h(x_\text{2}) \si{[kNm]}$,
			width=15.2cm,
			height=5cm,
			xmin=0, xmax=\k,
			xtick={\k},
			axis lines = left,
			axis x line = middle,
			every axis x label/.style={
				at={(ticklabel* cs:1)},
				anchor=west,
			},
			at={(0cm,-18cm)}
			]
			\addplot[name path=A,
				domain=0:\k, 
				smooth, 
				line width=2pt, 
				blue, 
				] {-6.25*x+0.625};
			\path [name path=B]
				(\pgfkeysvalueof{/pgfplots/xmin},0) --
				(\pgfkeysvalueof{/pgfplots/xmax},0);
			\addplot [pattern=vertical lines, pattern color=blue] fill between [of=A and B];
		\end{axis}
        \end{tikzpicture}
\end{center}
